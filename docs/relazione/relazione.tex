\documentclass[a4paper,11pt]{article}
\usepackage{latexsym}
\usepackage[italian]{babel}
\usepackage[utf8]{inputenc}
\usepackage{pdfsync}
\usepackage{moreverb}
\author{Alessio Caiazza, Cosimo Cecchi}
\title{CaptureMJPEG: a MotionJPEG library for Procesing}
\frenchspacing
\begin{document}
\maketitle
\tableofcontents

%\begin{abstract}
%  Cosa faremo noi qui?
%\end{abstract}
\section{Introduzione}
\label{sec:introduzione}
Scopo del progetto, forse va bene anche come abstract.

\section{Analisi}
\label{sec:analisi}
Analisi svolte sulle performance

\section{Manuale}
\label{sec:manuale}
Guida all'installazione ed utilizzo di CaptureMJPEG
\subsection{Installazione}
\label{sec:installazione}

\subsection{Guida}
\label{sec:guida}



\section{Sviluppo}
\label{sec:sviluppo}
Come continuare lo sviluppo
\subsection{Ottenere i sorgenti}
\label{sec:sorgenti}
Prima di scaricare i sorgenti è necessario installare
Mercurial\footnote{Mercurial può essere scaricato dal sito
 http://www.selenic.com/mercurial/}, 
per la gestione dei sorgenti ed Ant\footnote{Ant può essere scaricato
dal sito http://ant.apache.org}, 
per la gestione della compilazione.

Per ottenere i sorgenti eseguire la clonazione del repository
mercurial disponibile all'indirizzo
\texttt{http://dev.abisso.org/capturemjpeg} 
dopodiché creare una copia del file \texttt{user\_pref.xml.template}
con nome \texttt{user\_pref.xml}.

Il file contiene la configurazione di ant per il progetto, tutti i
valori di default vanno bene ad eccezione della ``property''
\texttt{processing-core} che deve essere corretta con la path completa
al file \texttt{core.jar} incluso nella propria installazione di Processing.
\begin{verbatim}
<property name="processing-core" 
    value="C:\Programmi\processing-0135-expert\lib\core.jar" />
\end{verbatim}

A questo punto è necessario eseguire il dowload delle librerie incluse
in CaptureMJPEG eseguendo il comando:
\begin{verbatim}
   ant download_deps
\end{verbatim}

Quindi è possibile generare l'intera cartella di installazione con
il comando:
\begin{verbatim}
   ant deploy 
\end{verbatim}
\begin{figure}
  \centering
\begin{boxedverbatim}

 hg clone http://dev.abisso.org/capturemjpeg capturemjpeg   
 cd capturemjpeg
 cp user_pref.xml.template user_pref.xml

\end{boxedverbatim}  
  \caption{Come ottenere i sorgenti da terminale}
  \label{fig:clone}
\end{figure}

\subsection{Classi utilizzate}
\label{sec:classi}
Forniamo ora una descrizione sommaria delle classi sviluppate per la
libreria, per una trattazione più approfondita si rimanda alla
documentazione JavaDoc disponibile online all'indirizzo 
http://capturemjpeg.lilik.it/doc/




\end{document}
